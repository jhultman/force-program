\documentclass[12pt]{article}
\usepackage[english]{babel}
\usepackage[utf8x]{inputenc}
\usepackage{amsmath}
\usepackage{amssymb}
\usepackage{mathtools}
\usepackage{graphicx}
\usepackage{fancyref}
\usepackage{fancyhdr}
\usepackage[colorinlistoftodos]{todonotes}

\pagestyle{fancy}
\lhead{Jacob Hultman}
\rhead{A physics problem.}

\newcommand{\HRule}{\rule{\linewidth}{0.5mm}}
\newcommand{\Lagr}{\mathcal{L}}
\DeclarePairedDelimiter{\norm}{\lVert}{\rVert}

\begin{document}

Consider a mass $m$ on a frictionless table subject to a piece-wise constant force

\begin{equation*}
	f(t) = f_i \text{ for } t \in [i - 1, i) \text{ and } \; i = 1, \ldots, 10.
\end{equation*}

\bigbreak

Let $y(t) = (x, v)$ denote the position and velocity of the mass. We'll use the notation $z_i = z(i)$ for $z \in \{x, v, y\}$ since with a piece-wise
constant force the problem discretizes. 

Suppose $y_0 = (0, 0)$ and we seek a force program $f_i$  which lands the mass in a desired state $y_d = (1, 0)$ at $ t = 10$. 

\bigbreak

\noindent Under the assumed initial rest, the velocity $v(t)$ can be written: $v = A_{vf} f$.

\bigbreak

\noindent Let's find the entries of $A_{vf}$. Consider the rows $a_i$ of $A_{vf}$. The entries in the first row $a_1$ tell us how the forces affect the velocity of the mass at $t = 1$ seconds. Clearly, forces $f_2, f_3, ...$ can have no effect on $v_1$, since at $t = 1$ they have not yet been applied. Newton's first law for a constant force says $\Delta v = a \Delta t = \frac{f}{m} \Delta t$, and so with $\Delta t = (i+1) - (i) = 1$ we have that $a_1 = (\frac{f_1}{m}, 0, ...)$.

\bigbreak

\noindent To find $a_2$, we reason from causality just as before. It's clear that $v_2$ cannot depend on $f_3, f_4, ...$. The dependency on $f_1$ and $f_2$ is very simple: it's additive. In other words, the velocity at time $t$ depends on the running sum of the forces applied so far (each multiplied by the duration of its application). 
And so we can write: 

\begingroup
\Large
\begin{equation*}
	v_i = \frac{1}{m} \Sigma_{j = 1}^{i} f_j, \text{ for } \; i = 1, \ldots, 10,
\end{equation*}
\endgroup

or in matrix form:

\begin{equation*}
	A_{vf} = \frac{1}{m}
	\begin{pmatrix}
	    f_1    &        &        &        \\
	    f_1    &  f_2   &        &        \\
	    \vdots & \vdots & \ddots &        \\
	    f_1    & f_2    & \hdots & f_{10} \\
	\end{pmatrix},
\end{equation*}

where the entries not shown are zero. 

\bigbreak

\noindent Similarly, the position can be written as a function of the velocities: $x = A_{xv} v$.

\bigbreak

\noindent As before, let's think about the first row $a_1$ of $A_{xv}$. 
Since the applied forces are piecewise constant, the velocities are piecewise linear. 
Under constant acceleration, displacement is simply determined by the average of the velocities at the two endpoints. 
Under initial rest, we have then 

\begin{equation}
x_1 = \frac{1}{2} v_1,
\end{equation}

and similarly

\begin{equation}
x_2 = x_1 + \frac{1}{2} (v_1 + v_2). 
\end{equation}

With a substitution, we have:
\begin{equation}
x_2 = v_1 + \frac{1}{2} v_2. 
\end{equation}

Carrying out the induction, we see:

\begin{equation}
x_i = \Sigma_{j=0}^{i - 1} v_j + \frac{1}{2} v_i.
\end{equation}

\bigbreak

And so the elements of $A_{xv}$ are simply: 

\bigbreak

$
A_{xv}(i, j) = 
\left\{
\begin{array}{ll}
      0 & i > j \\
      \frac{1}{2} & i = j \\
      1 & i < j \\
\end{array} 
\right.
$

\bigbreak

\noindent Finally form $A_{xf} = A_{xv} A_{vf}$.

Of particular interest to us is the final state 
$y = (x_{10}, v_{10})$. Letting 
$a_{xf} = A_{xf}(10, :)$ and 
$a_{vf} = A_{vf}(10, :)$, and defining:

\begin{equation}
	A \in \mathbb{R} ^ {2 \times 10} = 
	\begin{pmatrix}
	    a_{xf} \\[2pt]
	    a_{vf} \\[2pt]
	\end{pmatrix},
\end{equation}

observe $y = A f$.

\bigbreak

As it stands, finding a force program is a feasibility problem. Let's add a regularizer to reign in the solutions. With an $l_2$ penalty the problem takes the form:

\bigbreak

\begin{equation}
\begin{aligned}
& \underset{f}{\text{minimize}}
& & \frac{1}{2} f^T f \\
& \text{subject to}
& & A f = y_d
\end{aligned}
\end{equation}

This is nothing but the least-norm problem for an overdetermined system of equations. Let's solve it for practice. We introduce a Lagrangian, 

\begin{equation}
\Lagr = f^T f + \lambda ^ T (A f - y_d)
\end{equation},

and take first order conditions (FOCs):

\begin{subequations}
    \begin{align}
    \frac{\partial \Lagr}{\partial f} &= f + A^T \lambda = 0  \label{eq:foca} \\
    \frac{\partial \Lagr}{\partial \lambda} &= A f - y_d = 0, \label{eq:focb}
    \end{align}
\end{subequations}

Left-multiplying \fref{eq:foca} by $A$ we have:

\begin{equation}
Af - A A^T \lambda = 0, 
\end{equation}

and substituting from \fref{eq:focb} we have: 

\begin{equation}
A A^T \lambda = -y_d. \label{eq:AATLambda}
\end{equation}

Since $A$ is fat and full-rank, $A A^T$ is invertible, and so \fref{eq:AATLambda} becomes: 

\begin{equation}
\lambda = -(A A^T)^{-1} y_d.
\end{equation}

Substituting $\lambda$ back into \fref{eq:foca}, we obtain the optimal $f$, which we recognize as the least-norm solution:

\begin{equation}
f = A^T (A A^T)^{-1} y_d.
\end{equation}

While $l_2$ gives us traction analytically, other penalties give rise to 
solutions with very nice interpretations, and in some cases yield a convex
formulation. For instance, consider an $\norm{\cdot}_1$ penalty on the forces, 
which in a propulsion application would actually correspond closely to fuel 
consumption. An $\norm{\cdot}_\infty$ would penalize the max thruster.

\end{document}
