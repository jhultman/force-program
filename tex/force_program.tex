\documentclass[12pt]{article}
\usepackage[english]{babel}
\usepackage[utf8x]{inputenc}
\usepackage{amsmath}
\usepackage{amssymb}
\usepackage{mathtools}
\usepackage{graphicx}
\usepackage{fancyref}
\usepackage{fancyhdr}
\usepackage[colorinlistoftodos]{todonotes}

\pagestyle{fancy}
\lhead{Jacob Hultman}
\rhead{A physics problem.}

\newcommand{\HRule}{\rule{\linewidth}{0.5mm}}
\newcommand{\Lagr}{\mathcal{L}}
\DeclarePairedDelimiter{\norm}{\lVert}{\rVert}

\begin{document}

Consider a mass $m$ on a frictionless table subject to a piecewise constant force

\begin{equation*}
	f(t) = f_i \text{ for } t \in [i - 1, i) \text{ and } \; i = 1, \ldots, T_n.
\end{equation*}

\bigbreak

Let $y(t) = (x, v)$ denote the position and velocity of the mass. We'll use the notation $z_i = z(i)$ for $z \in \{x, v, y\}$ since with a piecewise
constant force the problem discretizes. Suppose $y_0 = \mathbf{0}$, $T_n = 10$ and we seek a force program $f_i$  which lands the mass in a desired state $y_d = (1, 0)$ at $t = T_n$. 

\bigbreak

\noindent Under the assumed initial rest, the velocity $v(t)$ can be written: $v = A^{vf} f$. Since the system is causal it is clear that $A^{vf}_{ij} = 0$ for $i < j$. (The velocity $v_i$ cannot depend on a force $f_j$ which has not yet been applied.) Applying Newton's first law under constant force gives $\Delta v = a \Delta t = \frac{f}{m} \Delta t$. For piecewise constant $f$ and $\Delta t = (i+1) - (i) = 1$, the velocity at time $t$ is simply the running sum of the forces applied so far:

\begin{equation*}
	v = A^{vf} f = \frac{1}{m}
	\begin{pmatrix}
	    1      &        &   \\
	    \vdots & \ddots &   \\
	    1      & \hdots & 1 \\
	\end{pmatrix} f,
\end{equation*}

where the upper-diagonal elements are zero. 

\bigbreak

Since the applied forces are piecewise constant, the velocities are piecewise linear, and the discrete-time positions can be written as a linear function of the sampled velocities: $x = A^{xv} v$, where

\bigbreak

\begin{equation*}
A^{xv}_{i, j} = 
\left\{
\begin{array}{ll}
      0 & i > j \\
      \frac{1}{2} & i = j \\
      1 & i < j \\
\end{array} 
\right.
\end{equation*}

\bigbreak

\noindent Next form $A^{xf} = A^{xv} A^{vf}$.

Of particular interest to us is the final state 
$y = (x_{10}, v_{10})$. Defining row vectors 
\begin{equation*}
    (a_i^{xf})^T = A^{xf}_{10, i} \quad\text{and}\quad 
    (a_i^{vf})^T = A^{vf}_{10, i}
\end{equation*}

and defining:

\begin{equation}
	A \in \mathbb{R} ^ {2 \times 10} = 
	\begin{pmatrix}
	    (a^{xf})^T \\[2pt]
	    (a^{vf})^T \\[2pt]
	\end{pmatrix},
\end{equation}

observe $y = A f$.

\bigbreak

As it stands, finding a force program is a feasibility problem. Let's add a regularizer to reign in the solutions. With an $l_2$ penalty the problem takes the form:

\bigbreak

\begin{equation}
\begin{aligned}
& \underset{f}{\text{minimize}}
& & \frac{1}{2} f^T f \\
& \text{subject to}
& & A f = y_d,
\end{aligned}
\end{equation}

\noindent which is the least-norm problem for an overdetermined system of equations. Let's solve it for practice. We introduce a Lagrangian, 

\begin{equation}
\Lagr = f^T f + \lambda ^ T (A f - y_d)
\end{equation}

\noindent and take first order conditions (FOCs):

\begin{subequations}
    \begin{align}
    \frac{\partial \Lagr}{\partial f} &= f + A^T \lambda = 0  \label{eq:foca} \\
    \frac{\partial \Lagr}{\partial \lambda} &= A f - y_d = 0. \label{eq:focb}
    \end{align}
\end{subequations}

Left-multiplying \fref{eq:foca} by $A$ we have:

\begin{equation}
Af - A A^T \lambda = 0, 
\end{equation}

and substituting from \fref{eq:focb} we have: 

\begin{equation}
A A^T \lambda = -y_d. \label{eq:AATLambda}
\end{equation}

Since $A$ is fat and full-rank, $A A^T$ is invertible, and so \fref{eq:AATLambda} becomes: 

\begin{equation}
\lambda = -(A A^T)^{-1} y_d.
\end{equation}

Substituting $\lambda$ back into \fref{eq:foca}, we obtain the optimal $f$, which we recognize as the least-norm solution:

\begin{equation}
f = A^T (A A^T)^{-1} y_d.
\end{equation}

While $l_2$ gives us traction analytically, other penalties give rise to 
solutions with very nice interpretations, and in some cases yield convex
formulations. In a propulsion application, an $\norm{\cdot}_1$ penalty on the forces would correspond closely to constraining fuel consumption (Boyd), while $\norm{\cdot}_\infty$ would penalize the max thruster.

\end{document}